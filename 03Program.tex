% Prof. Dr. Ausberto S. Castro Vera
% UENF - CCT - LCMAT - Curso de Ci\^{e}ncia da Computa\c{c}\~{a}o
% Campos, RJ,  2020
% Disciplina: Paradigmas de Linguagens de Programa\c{c}\~{a}o
% Aluno:


\chapterimage{ScalaH} % Chapter heading image ==>  Trocar este arquivo por outro 1200x468
\chapter{ Programa\c{c}\~{a}o em Scala}


%%%%%%%%======================
\section{Entradas e sa\'{\i}das}
%%%%%%%%======================
\begin{quote}
  Como já mencionado Scala é muito semelhante à linguagem Java e ambos usam Java Virtual Machine (JVM) para executar o código. Portanto, Scala está próximo da linguagem Java.
  \cite{Wampler2021}
  
  \hspace{2.5mm}Exemplo de saida de um Println:

  \begin{lstlisting}
  object Hallo{
    def main(args: Array[String]) {
      println("Hallo Welt!")
    }
  }
  \end{lstlisting}

  \hspace{2.5mm}Exemplo de saida e entrada Scala:


  \begin{lstlisting}
  object Entrada extends App {

      print("Primeiro nome: ")
      val pNome = readLine()

      print("Segundo nome: ")
      val sNome = readLine()

      println(s"Nome completo $pNome $sNome")

  }
  \end{lstlisting}

\end{quote}

%\subsection{Entrada e Sa\'{\i}da formatada}


%%%%%%%%======================
\section{Sele\c{c}\~{a}o}
%%%%%%%%======================
\begin{quote}
  Seleção é usado para tomar uma decisão com base nas condições. Um trecho de código é executado ou não é executado com base em uma condição especificada. Isso controla o fluxo de execução do programa permitindo decidir os passs seguintes.
  \cite{Wampler2021}
  \hspace{2.5mm} Exemplos de Seleção em Scala:

  \begin{itemize}
    \item if
    \item if-else
    \item Nested if-else
    \item if-else if ladder
  \end{itemize}

  \hspace{2.5mm}Exemplo de Instruções:

  \begin{itemize}

    \item if/else:

          \begin{lstlisting}
    if x < 10 then
      println("negative")
    else if x == 10 then
     println("10")
    else
      println("Nao eh igual a 10")
  \end{lstlisting}

    \item Nested if-else:

          \begin{lstlisting}

  object teste {

  def main(args: Array[String]) {

      var a: Int = 20
      var b: Int = 100
      var c: Int = 200

      if (a > b) // condition1
      {
          if(a > c) // condition2
          {
              println("o maior (a)");
          }

          else
          {
              println("o maior (c)")
          }

      }

      else
      {

          if(b > c)  // condition3
          {
              println("o maior (b)")
          }

          else
          {
              println("o maior (c)")
          }
      }
  }
  \end{lstlisting}
  \end{itemize}
\end{quote}





%%%%%%%%======================
\section{Repeti\c{c}\~{a}o}
%%%%%%%%======================
\begin{quote}
  Repetições são uma parte muito importante de qualquer linguagem de programação. Isso nos permite executar um conjunto de código várias vezes.
  \cite{Wampler2021}

  \hspace{2.5mm}Exemplos de repetições em Scala:

  \begin{itemize}

    \item While
    \item Do while
    \item For

  \end{itemize}

  \hspace{2.5mm}Exemplo de Instruções:

  \begin{itemize}
    \item for:


          \begin{lstlisting}
      var ints = List(1, 2, 3)
        for i <- ints do println(i)
      \end{lstlisting}

    \item while:

          \begin{lstlisting}
      var i = 0
        while(i < 11) {
          println(i)
          i += 1
        }
      \end{lstlisting}
  \end{itemize}
\end{quote}
%%%%%%%%======================
\section{Classes e Objetos}
%%%%%%%%======================
\begin{quote}
  As classes são projetos para a criação de objetos. Quando definimos uma classe, podemos então criar novos objetos (instâncias) da classe.
  \cite{Wampler2021}Portanto,
  \begin{itemize}
  \item Classe é um projeto de um objeto.
  \item Objeto nada mais é do que uma instância de uma classe.
  \end{itemize}
  \hspace{2.5mm} Exemplo de Classes e seus objetos em Scala:

  \begin{lstlisting}
  class Empregado(idc:Int,namec:String) {
      var ID:Int=idc
      var nome:String=namec

      def getNome(): String =
      {
         nome
      }

  }

  object Main {
    def main(args: Array[String]): Unit = {
      val empr1:Empregado =new Empregado(23,"Daniel")
      val empr2:Empregado =new Empregado(23,"Terra")
      print(emp1.getNome())

    }
  }
  \end{lstlisting}

\end{quote}

%%%%%%%%======================
\section{Fun\c{c}\~{o}es}
%%%%%%%%======================
\begin{quote}
  Scala também oferece suporte a uma abordagem de programação funcional. Portanto, possui funções integradas e também permite criar funções personalizadas.
  Funções são valores principais em Scala. Dessa forma, é possível armazenar o valor de uma função, passar uma função como um argumento e retornar uma função como um valor de outra função. Tornando possível o uso de POO - (Programação Orientada a Objeto).
  \cite{Wampler2021}

  \begin{itemize}
    \hspace{2.5mm} Exemplo de Syntax de função em Scala:

    \begin{lstlisting}
  def funcNome(parameters : type) : return type = {
   // corpo da funcao
  }
  \end{lstlisting}


  \hspace{2.5mm} Exemplo de função em Scala:

  \begin{lstlisting}
    object Main {
      def printFunc(n:Int): Unit ={
        if(n<0)
        return ;
        println(n)
        printFun(n-1)
        }

        def main(args: Array[String]): Unit = {
          printFun(3);
          }
          }
  \end{lstlisting}

\end{itemize}
\end{quote}



%%%%%%%%======================
\section{Herança}
%%%%%%%%======================
\begin{quote}
Herança é um pilar importante da OOP (Programação Orientada a Objetos). É o mecanismo em Scala pelo qual uma classe pode herdar as funções de outras classes.
\cite{Wampler2021}

\hspace{2.5mm}Informações Importantes:

\begin{itemize}

\item Superclasse: a classe cujas características são herdadas é conhecida como superclasse ou uma classe base ou uma classe pai.

\item Subclasse: a classe que a outra classe herda é chamada de subclasse ou uma classe derivada, classe estendida ou classe filha. A subclasse pode adicionar seus próprios campos e métodos, além dos campos e métodos da superclasse.

\item Reutilização: a herança apóia o conceito de polimofismo;  se quisermos criar uma nova classe e já houver uma classe que contenha parte do código que desejamos, podemos derivar nossa nova classe da classe existente. Desta forma, reutilizamos os campos e métodos da classe existente.

\end{itemize}

\hspace{2.5mm} Exemplo da Syntax de Herança em Scala:

\begin{lstlisting}
class parent_class_name extends child_class_name{
// metodos etc
}
\end{lstlisting}

\hspace{2.5mm} Exemplo de Herança em Scala:

\begin{lstlisting}

class Aluno{
    var Name: String = "Daniel"
}

class Aluno1 extends Aluno
{
    var num: Int = 20

    def details()// Metodo
    {
    println("Nome aluno: " +Nome);
    println("Total de materias: " +num);
    }
}

object Main
{

    def main(args: Array[String])
    {

        val ts = new Aluno1();
        ts.details();
    }
}
\end{lstlisting}



\end{quote}
