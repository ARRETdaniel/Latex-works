\documentclass{article}
\usepackage[brazil]{babel}

\usepackage[a4paper,top=2cm,bottom=2cm,left=3cm,right=3cm,marginparwidth=1.75cm]{geometry}
% Useful packages
\usepackage{amsmath}
\usepackage{graphicx}
\usepackage[colorlinks=true, allcolors=blue]{hyperref}
\usepackage{minted}
\usepackage{float}
\usepackage{soul}
\usepackage{booktabs}
\usepackage{graphicx}
\usepackage[table,xcdraw]{xcolor}

\title{Plano de trabalho 2022}
\author{Daniel Terra Gomes}

\begin{document}
\begin{titlepage}
\begin{center}
\large
\textbf{PROGRAMA INSTITUCIONAL DE BOLSAS DE INICIA\c{C}\~{A}O CIENTIF\'{I}CA E TECNOL\'{O}GICA\\\vspace{0,5cm}
UNIVERSIDADE ESTADUAL DO NORTE FLUMINENSE DARCY RIBEIRO\\
}
\textit{Centro CCT \\
Labotat\'{o}rio LCMAT\\
\vspace{1cm}}
%Plano de trabalho para o segundo ano da bolsa }\\
\vspace{1,5cm}
\textbf{Plano de Trabalho para Bolsa de Iniciação Científica}\\\vspace{5cm}
\end{center}
\textbf{Bolsista}: Daniel Terra Gomes\\
\textbf{Matricula}: 00119110484\\
\textbf{Orientadora}: Prof. Dra. Annabell Del Real Tamariz  \\
\textbf{Curso}: Bacharelado em Ci\^{e}ncia da Computa\c{c}\~{a}o\\
\vspace{3cm}
\begin{center}
\textbf{Titulo do Projeto}: Project-driven Data Science: Aprendendo e Mapeando\\
\textbf{T\'{\i}tulo do Plano de Trabalho}: Veículos Autônomos no Brasil e suas tecnologias.
%Os principais obstáculos para Veículos Autônomos no Brasil, e suas tecnologias essenciais.\\

\textbf{Fonte financiadora:} PIBICT/UENF
\end{center}
\end{titlepage}


%\maketitle
\section{Justificativa}

Veículos são partes essenciais de nossas vidas, fazemos uso para ir a universidade, trabalho, escola, compras, viagens e muito mais. Sendo um dos principais meios de transporte em nossa sociedade.
Todavia, com a evolução tecnológica buscamos maneiras de tornar nossas vidas mais práticas, e automatizadas. A partir dessa necessidade surgem os veículos autônomos, que são veículos que dispensam parcialmente ou totalmente a exigência de um condutor para controlá-lo. Assim trazendo uma maior praticidade, segurança e conforto para os deslocamentos. Sendo a segurança e praticidade um dos principais pontos para aqueles que desejam aderir a tecnologia.
Estima-se que no brasil o número de mortes em acidentes de transporte terrestre no período de 2019 foi de 31.945 \cite{Anexo_I_pnatrans}. Veículos autônomos vêm com a promessa de buscar uma redução nesses números através da retirada do principal causador de acidentes de trânsito: erros humanos. 

Ademais, Veículos Autônomos vem como uma forma de minimizar os congestionamentos nas grandes metrópoles. Segundo o  Plano Estratégico de Desenvolvimento Urbano Integrado da Região Metropolitana do Rio de Janeiro, \cite{rj_transito}, apenas na hora do rush da manhã o fluxo de viagens de São Gonçalo a Niterói chega a quase 100 mil pessoas sendo transportadas; desses deslocamentos cerca de 80\% das viagens são feitas em transporte público – ônibus convencionais. Diante disso, uma das propostas para suprir essa demanda de transporte seria a inclusão de veículos autônomos. Nesse formato, carros poderiam ser solicitados como, hoje, são feitas as corridas de aplicativos, e os ônibus do transporte público  poderiam operar por mais horas e com menor custo. 
Entretanto, ainda seria necessário lidar com outros problemas como a disputa de espaço nas vias, e os engarrafamentos crônicos das cidades; De acordo com informações do levantamento domiciliar realizado durante a elaboração do último PDTU, o tempo médio de deslocamento do centro de São Gonçalo a Niterói é de 50 minutos, devido a problemas relacionados ao grande fluxo de veículos, sendo o transporte público quase 25\% maior.

Portanto, podemos analisar que com a adesão desse tipo de transporte autônomo pela sociedade. Além de mais segurança trará uma diminuição no custo das viagens, e no tempo gasto. A partir de veículos interconectados e inteligentes, que podem se comunicar uns com os outros em tempo real. Possibilitando que os veículos  mantenham uma frequência de velocidade nas vias e evitando as trocas de faixas desnecessárias. Além disso, será possível escapar de áreas com maior número de veículos, e buscar rotas pela cidade que não tenham acidentes.


Sendo assim, neste primeiro ano propomos realizar um estudo bibliográfico para entender e mapear as perspectivas de Veículos Autônomos no Brasil, fazendo o paralelo com o que há de melhor na área no mundo, além do mais buscamos estruturar as tecnologias que são usadas para o funcionamento desses veículos. Desse modo, obtendo um aprendizado em diversas áreas e conceitos ligados à Inteligência Artificial que é uma tecnologia amplamente utilizada na área.
Por consequência, ao final deste ano teremos percorrido e estruturado o cenário de Veículos Autônomos no Brasil, e entendido as tecnologias para a sua operação. Assim alcançando uma aprendizagem eficaz, autodidata, e exploratória.



 


\section{Objetivos}
\begin{enumerate}
    \item  Entender o cenário de Veículos Autônomos no mundo, e contrastar com o brasileiro:
    \begin{enumerate}
        \item Compreender o cenário automobilístico brasileiro, e as suas expectativas para essa tecnologia.
        \item Contrastar o mercado de veículos autônomos mundial com o brasileiro, buscando  decifrar o que é necessário para a aplicação dessa tecnologia no país.
    \end{enumerate}
    \item  Estudar as principais empresas de pesquisa que trabalham com Veículos Autônomos no mundo, e o que buscam economicamente e tecnologicamente no setor:
    \begin{enumerate}
        \item Identificar se buscam diferentes tipos de Carros Autônomos. Assim como entender as suas possíveis principais diferenças.
        \item Entender o que essas empresas buscam alcançar economicamente, e tecnologicamente ao inserir essa tecnologia no mercado.
        \item Conhecer as mudanças econômicas que carros autônomos podem trazer para a sociedade brasileira. 
    \end{enumerate}
    \item Mapear as tecnologias essenciais para a Direção Autônoma:
    \begin{enumerate}
        \item Documentar quais são os Softwares, algoritmos de controle, e sensores usados nesses veículos.  
    \end{enumerate}
\end{enumerate}

\section{Metodologia}
Baseado no “Project-based learning” \cite{krajcik2006project}. Seguiremos os estudos através de um projeto que aborda problemas do mundo real, cujo muitos não tem resposta única. Ao longo desse projeto será possível fazer novas perguntas e encontrar suas possíveis respostas por meio de uma investigação sustentada.


\vspace {1mm}

Este Plano de Pesquisa também utilizará as seguintes metodologias:
\begin{itemize}
\item \textit{Pesquisa bibliográfica: sobre os diferentes assuntos relacionados com Inteligencia Artificial, Machine Learning, Veículos Autônomos;
}
\item \textit{Seminários e minicursos;}
\item \textit{Participação em eventos;}
\item \textit{Publicação de resultados.}

\end{itemize}


\subsection{Veículos Autônomos no Brasil e no mundo}

Nesta fase, iremos entender e fazer um estudo bibliográfico das iniciativas e expectativas do Brasil e do mundo para essa tecnologia. Atrelado a isso, faremos uma análise para identificar quais são os fatores necessários para aplicação dessa tecnologia pelo o mundo, sobretudo, no Brasil. Durante essas pesquisas iremos bibliografar os achados assim formando uma mapa estruturado das principais pesquisas e trabalhos na área.


\subsection{Veículos Autônomos e suas perspectivas}

O entendimento das perspectivas sociais e econômicas de uma tecnologia é vital para que possamos alocar recursos, e gerar mão de obra qualificada para o desenvolvimento de Tecnologias Disruptivas \cite{4cenarios_ocidental}, cujo sao inovações que são responsáveis por trazer grandes mudanças para o mercado e impulsionar tecnologicamente a sociedade, assim atendendo necessidades futuras da sociedade. 
Dessa forma,  analisaremos acervos, pesquisas, e projetos. A fim de identificar as tendências econômicas, tecnológicas e sociais dessa tecnologia.


\subsection{Tecnologias Essenciais para a Direção Autônoma}
Nesta etapa, iremos mapear as tecnologias usadas para o desenvolvimento de um veículo autônomo. Com o intuito de compreender os seus recursos fundamentais e funcionalidades nesse tipo de veículo.  Portanto, realizaremos um estudo bibliográfico com o intuito de entender os principais algoritmos, software e artifícios físicos (Hardware) usados em Carros Autônomos. Dessa forma, se tornaram nítidos quais são os recursos e conhecimentos necessários para o desenvolvimento e implementação desses veículos no país. Neste ano, propomos como ponto de partida a leitura do livro \cite{aurelien2017hands}, buscando entender cada um dos modelos apresentados referentes ao campo de Veículos Autônomos.

\section{Objetivo em Etapas} \label{etapas}
A fim de alcançar os objetivos do Projeto de Pesquisa, neste Plano listamos as principais atividades que serão realizadas durante o período e vigência da bolsa:


\begin{enumerate}
    \item  Estudo bibliográfico das perspectivas nacional e internacional no que diz respeito a veículos autônomos;
    \item  Pesquisa bibliográfica para compreender o que busca economicamente e tecnologicamente o mercado internacional e nacional em relação a veículos autônomos;
    \item Aprender quais são os diferentes tipos de veículos autônomos;
    \item Pesquisa bibliográfica das tecnologias essenciais de um carro autônomo;
    \item Mapear e entender os principais softwares de controle de um carro autônomo;
    \item Elaboração do Relatório Final.
\end{enumerate}

\section{Cronograma das atividades}
Este cronograma visa mostrar o desenvolvimento de atividades (listadas na Seção \ref{etapas}), cada etapa foi dividido de modo a otimizar o tempo e as necessidades do projeto.

% Please add the following required packages to your document preamble:
% \usepackage{booktabs}
% \usepackage{graphicx}
% \usepackage[table,xcdraw]{xcolor}
% If you use beamer only pass "xcolor=table" option, i.e. \documentclass[xcolor=table]{beamer}
\begin{table}[H]
\centering
\resizebox{\textwidth}{!}{%
\begin{tabular}{@{}l|l|l|l|l|l|l|l|l|l|l|l|l|@{}}

\cmidrule(l){1-13}


\multicolumn{1}{c|}{\textbf{Etapas/Mês}} &

  \multicolumn{1}{c|}{\textbf{1º}} &
  \multicolumn{1}{c|}{\textbf{2º}} &
  \multicolumn{1}{c|}{\textbf{3º}} &
  \multicolumn{1}{c|}{\textbf{4º}} &
  \multicolumn{1}{c|}{\textbf{5º}} &
  \multicolumn{1}{c|}{\textbf{6º}} &
  \multicolumn{1}{c|}{\textbf{7º}} &
  \multicolumn{1}{c|}{\textbf{8º}} &
  \multicolumn{1}{c|}{\textbf{9º}} &
  \multicolumn{1}{c|}{\textbf{10º}} &
  \multicolumn{1}{c|}{\textbf{11º}} &
  \multicolumn{1}{c|}{\textbf{12º}} \\ \midrule

 
\multicolumn{1}{|l|}{\textbf{1}} 

\cellcolor[HTML]{A4C2F4}&
   &
  \cellcolor[HTML]{A4C2F4} &
  \cellcolor[HTML]{A4C2F4} &
   &
   &
   &
   &
   &
   &
   &
   &
   \\ \midrule
\multicolumn{1}{|l|}{\textbf{2}} &
   &
   &
  \cellcolor[HTML]{A4C2F4} &
  \cellcolor[HTML]{A4C2F4} &
  \cellcolor[HTML]{A4C2F4} &
   &
   &
   &
   &
   &
   &
   \\ \midrule
\multicolumn{1}{|l|}{\textbf{3}} &
   &
   &
   &
   &
   \cellcolor[HTML]{A4C2F4}&
   \cellcolor[HTML]{A4C2F4}&
   \cellcolor[HTML]{A4C2F4}&
   &
   &
   &
   &
   \\ \midrule
\multicolumn{1}{|l|}{\textbf{4}} &
   &
    \cellcolor[HTML]{A4C2F4}&
   &
  \cellcolor[HTML]{A4C2F4} &
   &
   &
  \cellcolor[HTML]{A4C2F4} &
   &
   &
  \cellcolor[HTML]{A4C2F4} &
   &
   \\ \midrule
\multicolumn{1}{|l|}{\textbf{5}} &
   &
   &
  \cellcolor[HTML]{A4C2F4} &
   &
   &
  \cellcolor[HTML]{A4C2F4} &
   &
   \cellcolor[HTML]{A4C2F4}
   &
   \cellcolor[HTML]{A4C2F4}
   &
  \cellcolor[HTML]{A4C2F4} 
  &
  \cellcolor[HTML]{A4C2F4} 
  &
  \cellcolor[HTML]{A4C2F4} 
   \\ \midrule
\multicolumn{1}{|l|}{\textbf{6}} &
   &
   &
   &
   &
   &
  \cellcolor[HTML]{A4C2F4} &
   &
  \cellcolor[HTML]{A4C2F4} &
   \cellcolor[HTML]{A4C2F4}&
   &
   &
  \cellcolor[HTML]{A4C2F4} \\ 
  \bottomrule
  
\end{tabular}%
}
\caption{Etapas do plano de trabalho}
\label{tab:etapas}
\end{table}


\bibliographystyle{alpha}
\bibliography{bibli}
\end{document}

