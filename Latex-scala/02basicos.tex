% Prof. Dr. Ausberto S. Castro Vera
% UENF - CCT - LCMAT - Curso de Ci\^{e}ncia da Computa\c{c}\~{a}o
% Campos, RJ,  2020
% Disciplina: Paradigmas de Linguagens de Programa\c{c}\~{a}o
% Aluno:


\chapterimage{ScalaH} % Chapter heading image ==>  Trocar este arquivo por outro 1200x468
\chapter{ Conceitos b\'{a}sicos da Linguagem Scala}

\begin{quote}
  Iniciando na linguagem Scala. Como já introduzido no capítulo anterior, Scala é bem similar a linguagem Java, sabendo que as duas fazem uso do Java Virtual Machine(JVM) para fazer a execução dos códigos desenvolvidos. Portanto, para os leitores desse material que já são  familiarizados com a linguagem Java não terão dificuldades para entender como se desenvolve em Scala.
  \cite{Wampler2021}
\end{quote}
%%%%%%%%=================================
\section{Vari\'{a}veis e constantes}
%%%%%%%%=================================
\begin{quote}
  Variáveis reservam lugar em memória para serem usados durante a execução do programa. A linguagem Scala possui dos tipos de variáveis; "Imutáveis" e “mutáveis”.
  Sendo, imutáveis variáveis que não podem ter o seu valor alterado durante a execução do programa, esse tipo de variável é declarada com a palavra chave {\color{red}(val)}. Por outro lado, mutáveis podem ser alteradas durante a execução, declaradas com {\color{red}(var)}.
  \cite{Wampler2021}
\end{quote}
\begin{quote}
  \hspace{2.5mm} Exemplos de variáveis imutáveis e mutáveis:

  \begin{lstlisting}
    object Main {
  def main(args: Array[String]) {
      val b: Int = 230 // Imutaveis
      var c: Long = 9223372036854775807 // mutaveis
  }
}
    \end{lstlisting}

\end{quote}
%%%%%%%%=================================
%\vspace{20mm} %%%%%%%%%%%%%%%%%%%distance between section
\section{Palavras reservadas}
%%%%%%%%=================================
\begin{quote}
  São um conjunto de palavras ou caracteres que são de uso exclusivo da linguagem. Essas palavras chaves não podem ser usadas como nome de variáveis.
  \cite{Wampler2021}

\end{quote}
%%\begin{itemize}
\begin{quote}
  \hspace{2.5mm} Palavras chaves em Scala:

  \begin{tabular}{ |c|c|c|c| }
    \hline
    \multicolumn{3}{|c|}{Palavras chaves} \\
    \hline
    abstract  & throw   & try             \\
    \hline
    class     & catch   & super           \\
    \hline
    override  & private & extends         \\
    \hline
    for       & import  & null            \\
    \hline
    this      & case    & trait           \\
    \hline
    protected & return  & sealed          \\
    \hline
    true      & type    & val             \\
    \hline
    while     & with    & yield           \\
    \hline
    def       & do      & else            \\
    \hline
    false     & final   & finally         \\
    \hline
    forSome   & if      & implicit        \\
    \hline
    lazy      & match   & new             \\
    \hline
    object    & var     & package         \\
    \hline
    -         & :       & =               \\
    \hline
    <-        & </      & <:              \\
    \hline
    % @ & # & =>	\\
    \hline
    >:        & @       & =>              \\
    \hline
  \end{tabular}

  %%\end{itemize}
\end{quote}
%%%%%%%%=================================
\section{Tipos de Dados B\'{a}sicos}
%%%%%%%%=================================

\begin{quote}
  %%%%%%%%=================================
  Os tipos de dados básicos da linguagem Scala são objetos e são categorizados da seguinte forma: Int, Byte, Short,  Long, Float, Double, Char, String, Boolean, Unit, Null, Nothing, AnyRef, Any.
  Sabendo que, os tipos de dados em Scala não tem tipo primitivo como no Java. Portanto, é possível chamar métodos em Long, Int, etc.
  \cite{Wampler2021}
\end{quote}

\begin{quote}
  \hspace{2.5mm} Tipo de dados em Scala:

  \begin{tabular}{ |c|c|c| }
    \hline
    Tipo de Dado & Descrição                                                         \\
    \hline
    Int          & 32 bit signed value. Tamanho -2147483648 até 2147483647           \\
    \hline
    Boolean      & true ou false                                                     \\
    \hline
    Byte         & 8 bit signed value. Tamanho -128 até 127                          \\
    \hline
    Short        & 16 bit signed value. Tamanho -32768 até 32767                     \\
    \hline
    Int          & 32 bit signed value. Tamanho -2147483648 até 2147483647           \\
    \hline
    Long         & 64 bit signed value. -9223372036854775808 ate 9223372036854775807 \\
    \hline
    Float        & 32 bit IEEE 754 single-precision float                            \\
    \hline
    Double       & 64 bit IEEE 754 double-precision float                            \\
    \hline
    Char         & 16 bit unsigned Unicode character. Tamanho de U+0000 to U+FFFF    \\
    \hline
    String       & Sequencia de caracteres                                           \\
    \hline
    Unit         & Sem valores correspondente                                        \\
    \hline
    Null         & null ou referência vazia                                          \\
    \hline
    Nothing      & Sub-tipo de qualquer outro valor; incluindo sem valores           \\
    \hline
    Any          & Sub-tipo de qualquer outro valor; Qualquer objeto é do tipo An    \\
    \hline
  \end{tabular}
\end{quote}

\begin{quote}
  \hspace{2.5mm} Exemplos de tipo de dados em Scala:

  \begin{lstlisting}
    object Main {
  def main(args: Array[String]) {
      val a: Byte = 4
      val b: Int = 230
      val c: Long = 9223372036854775807
      val d: Short = -32768
      val e: Double = 742.0
  }
}
    \end{lstlisting}

  Importante mencionar que se o tipo de dado não for mencionado o compilador ira detectá-lo automaticamente.
\end{quote}


%%%........................
%\subsection{String}

%%%%%%%%=================================
\section{Operadores e Express\~{o}es em Scala}
%%%%%%%%=================================
\begin{quote}
  Tudo em Scala pode se dizer uma expressão apenas {\color{red}(val)} e {\color{red}(class)} que não. Podemos dizer que uma expressão retornará um valor. Normalmente, é evitado fazer uso de Loops na linguagem Scala. Pois pode trazer problemas ao programa se mal implementado.
  \cite{Wampler2021}
\end{quote}

\begin{quote}
Exemplos de expressões em Scala:
\begin{itemize}

  \item Expressão If:

\begin{lstlisting}
val someCondition = true
val conditionedValue = if(someCondition) 10 else 5
\end{lstlisting}

  \item Expressão if/else:

Uma outra forma de se usar condicional {\color{red}(if)}:

\begin{lstlisting}
  if x < 10 then
    println("negative")
  else if x == 10 then
   println("10")
  else
    println("Nao eh igual a 10")
\end{lstlisting}

\item Expressão Loop While:

\begin{lstlisting}
var i = 0
  while(i < 11) {
    println(i)
    i += 1
  }
\end{lstlisting}

\item Expressão Loop for:


\begin{lstlisting}
var ints = List(1, 2, 3)
  for i <- ints do println(i)
\end{lstlisting}


\item Expressão Match:

\begin{lstlisting}
val a = Person("Daniel")

a match
  case Person(name) if name == "Daniel" =>
    println(s"$name says, Name found")

  case Person(name) if name == "Daniel T." =>
    println(s"$name says, Name found!")
\end{lstlisting}

\item Expressão  try/catch/finally:
\begin{lstlisting}
try
  arquivoEscrita(text)
catch
  case nho: IOException => println("Palavras com
  final nho.")
  case lho: NumberFormatException => println("Pala
  vras com final lho.")
finally
  println("Fim").

\end{lstlisting}





\end{itemize}
\end{quote}
%%%%%%%%%%%%%%%%%%%%%%%%%%%%%%%%%%
\begin{quote}
  %\hspace{2.5mm}
Operadores condicionais:

  \begin{tabular}{ |c|c|c| }
    \hline
    Operador & Operação       & descrição                                           \\
    \hline
    ee       & e              & Os dois valores da operação são verdadeiros.        \\
    \hline
    ||       & ou             & Pelo menos um dos valores é verdadeiro.             \\
    \hline
    >        & maior que      & O valor da esquerda é maior que o da direita.       \\
    \hline
    >=       & maior ou igual & O valor da esquerda é maior ou igual ao da direita. \\
    \hline
    <        & menor que      & O valor da esquerda é menor que o da direita.       \\
    \hline
    <=       & menor ou igual & O valor da esquerda é menor ou igual ao da direita. \\
    \hline
    ==       & igual          & Os dois valores são iguais.                         \\
    \hline
    !=       & diferetens     & Os valores são diferentes.                          \\
    \hline
  \end{tabular}
\end{quote}
