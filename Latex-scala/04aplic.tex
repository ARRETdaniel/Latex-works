% Prof. Dr. Ausberto S. Castro Vera
% UENF - CCT - LCMAT - Curso de Ci\^{e}ncia da Computa\c{c}\~{a}o
% Campos, RJ,  2020
% Disciplina: Paradigmas de Linguagens de Programa\c{c}\~{a}o
% Aluno:



\chapter{ Aplica\c{c}\~{o}es da Linguagem Scala}
\begin{quote}
  A seguir serão apresentados um será de aplicações com a Linguagem Scala. Dessa forma, será possível ver na prática o que é possível ser feito na linguagem.
  \cite{alexander2013scala}
  \hspace{2.5mm}Entre os exemplos estão os seguintes:
  \begin{itemize}
    \item Operações básicas aritmética;
    \item O algoritmo Quicksort;
    \item Programa de Cálculo Numérico, equação linear;
    \item Aplicação usando Matrizes
    \item Aplicações Profissionais da linguagem de Linear Regression usando Spark.
  \end{itemize}



\end{quote}

%%%--------------------------------------------------------------------
\section{Opera\c{c}\~{o}es b\'{a}sicas}
%%%--------------------------------------------------------------------
\begin{quote}

  \hspace{2.5mm}Exemplo algoritmo operador aritmético:
  \cite{alexander2013scala}
  \begin{lstlisting}
      object Atitmetico {
      // criando o objeto
        def main(args:Array[String]) {
          var x = 50;
          var y = 10;
            // soma de x + y
            println("Add of x + y = " + (x + y));
            // Subitracao entre x e y
            println("Sub of x - y = " + (x - y));
            // Multiplicacao dos valores apresentados
            println("Multi of x * y = " + (x * y));
            // Divisao x e y
            println("Divisao of x / y = " + (x / y));
            // Modulo dos valores
            println("Mod of x % y = " + (x % y));

        }
      }
    \end{lstlisting}

\end{quote}
%%%--------------------------------------------------------------------
\section{O algoritmo Quicksort em Scala}
%%%--------------------------------------------------------------------
\begin{quote}
  \hspace{2.5mm}Exemplo algoritmo Quicksort em Scala:
  \cite{alexander2013scala}

  \begin{lstlisting}

    // Implementa um exemplo para quicksort.
    // O algoritmo funciona da seguinte maneira:
    // Mova da esquerda para a direita todos os
    // valores
    // que sao menores que
    // o valor pivo para o
    // a esquerda de uma divisao flutuante, em que
    // a divisao
    // sempre marca
    // o ponto para
    // a esquerda disso, todos os valores sao menores
    // que o pivo.
    // Mova-se no
    // final
    // o valor pivo para a posicao de divisao.
    // Em seguida,
    // recursao para
    // os subarrays
    // em ambos os lados do ponto de pivo.



   object Quicksort {
     def main(args: Array[String]) {
       var mess = Array(3, 9, 8, 13, 2, 5, 4);

       quicksort(mess, 0, mess.length-1);
       mess.foreach( println )
     }

     // Troca dois valores em uma matriz
     def swap(a: Array[Int], pos1: Int, pos2: Int):
      Unit = {
       val stash = a(pos1)
       a(pos1) = a(pos2)
       a(pos2) = stash
     }

     // Executa a classificacao rapida recursiva
     // em uma matriz
     // Existe uma opcao para limitar a complexidade
     // do espaco
     // para
     // O (ln (n)) aqui, quando da ultima vez recursiva
     // na
     // particao maior
     // (Recursao da cauda).

     def quicksort(a: Array[Int], low: Int, hi: Int):
      Unit = {
       if (low < hi) {
         val p = partition(a, low, hi)
         quicksort(a, low, p-1)
         quicksort(a, p+1, hi)
       }
     }

     // Particionar uma matriz em torno de um pivo
     // Seleciona um pivo, desloca todos os valores
     // menores que
     // o pivo para a esquerda e desloca todos
     // os valores
     // maior do que o ponto de pivo a direita dele.
     //
     // Para otimizacao: escolha um pivo de acordo
     // com uma
     // mediana
     // Valor de uma amostra. Isso ajuda que
     // O particionamento leva a submatrizes de
     // tamanhos
     // semelhantes,
     // e, portanto, o desempenho ideal de
     // O (n // ln (n)).
     //
     // Indice de retorno (localizacao) do pivo
     // na matriz
     // particionada

     def partition(subArray: Array[Int], low: Int,
      hi: Int):
       Int = {
       val pivot = hi;
       var i = low;
       for (
         j <- low to hi
         if subArray(j) < subArray(pivot)
       ) {swap(subArray, i, j); i+=1}

       // Finalmente, mova o valor pivo para a linha
       // divisoria
       // em i
       swap(subArray, i, pivot);
       return i
     }
   }

  \end{lstlisting}

\end{quote}

%%%--------------------------------------------------------------------
\section{Programa de C\'{a}lculo Num\'{e}rico}
%%%--------------------------------------------------------------------
\begin{quote}
  Os seguintes exemplos fazem uso da Biblioteca Apache Commons Math é um projeto
  Apache que visa resolver os problemas matemáticos e estatísticos mais comuns
  que não estão disponíveis na linguagem Java padrão. Ele suporta classes de matriz
  densa e esparsa que são equipadas com operações básicas, bem como algoritmos de
  decomposição de matriz. \cite{alexander2013scala}

  \hspace{2.5mm}Exemplo algoritmo calculando uma equação linear:

  \begin{lstlisting}
    // Import biblioteca
    import org.apache.commons.math3.linear._

    // Crie a matriz
    val A = new Array2DRowRealMatrix(3, 6)

    // Preencha a matriz com numeros aleatorios
    val r = new scala.util.Random(0)

    for(i <- 0 until A.getRowDimension())
        for(j <- 0 until A.getColumnDimension())
            A.setEntry(i, j, r.nextDouble())

    // Defina o primeiro valor como o ultimo valor
    A.setEntry(0, 0,
        A.getEntry(A.getRowDimension() - 1, A.
          getColumnDimension() - 1))

    // Obtenha uma submatriz, 1 linha a 3 linha,
    // 1 coluna a 3 coluna
    val B = A.getSubMatrix(0, 2, 0, 2)

    // Resolva a equacao linear
    val solver = new LUDecomposition(B).getSolver()
    val a = A.getColumnVector(0)
    val x = solver.solve(a)
   \end{lstlisting}
\end{quote}
%%%--------------------------------------------------------------------
\section{Aplica\c{c}\~{a}o usando Matrizes}
%%%--------------------------------------------------------------------
\begin{quote}
  \hspace{2.5mm}Exemplo algoritmo calculo de produto de matrizes:
  \cite{alexander2013scala}
  \begin{lstlisting}
    // Obtenha uma submatriz, 1 linha a 3
    // linha, 1 coluna a
    // 3 coluna
    val B = A.getSubMatrix(0, 2, 0, 2)

    // Defina uma sub-matriz de A a B, da
    // 1 linha a 3 linha,
    // 2 coluna a 4 coluna
    A.setSubMatrix(B.getData(), 0, 1)

    // Produto de matriz C = A'B
    val C = A.transpose().multiply(B)

  \end{lstlisting}
\end{quote}
%%%--------------------------------------------------------------------
\section{Aplica\c{c}\~{o}es Profissionais}
%%%--------------------------------------------------------------------

\begin{quote}
  \hspace{2.5mm} Aplicações de Linear Regression usando Spark desenvolvido em Scala.

  Nessa aplicação iremos fazer um modelo para predizer a nota de alunos com base na sua nota anterios e seu tempo de estudo.
  \cite{9781787122383}
  \begin{lstlisting}
    //Instale todas as dependencias no ambiente
    //Apache Spark
    //2.4.4 with hadoop 2.7, Java 8 e Findspark
    //para localizar o
    //Spark no sistema
    !apt-get install openjdk-8-jdk-headless -qq > /
    dev/null
    !wget -q http://apache.osuosl.org/spark/
    spark-2.4.4/
    spark-2.4.4-bin-hadoop2.7.tgz
    !tar xf spark-2.4.4-bin-hadoop2.7.tgz
    !pip install -q findspark

    // Variaveis de ambiente
    import os
    os.environ["JAVA_HOME"] =
    "/usr/lib/jvm/java-8-openjdk-amd64"
    os.environ["SPARK_HOME"] =
    "/content/spark-2.4.4-bin-hadoop2.7"

    //Iniciar sessao do Spark
    import findspark
    findspark.init()
    from pyspark.sql import SparkSession
    spark =
    SparkSession.builder.master("local[*]").getOrCreate
    ()
    //Carregar o arquivo Student_Grades_Data.csv do
    //sistema local para o local remoto da colab
    from google.colab import files
    files.upload()

    //Carregando o arquivo Student_Grades_Data.csv,
    // carregado na etapa anterior
    data = spark.read.csv('Student_Grades_Data.csv',
     header=True, inferSchema=True)

    //Dando uma olhada no tipo de dados de cada
    //coluna para ver quais tipos de dados o parametro
    //inferSchema = TRUE definiu para cada coluna
    data.printSchema()

    //root
    // |-- Time_to_Study: integer (nullable = true)
    // |-- Grades: double (nullable = true)

    //Exibir as primeiras linhas de dados
    data.show()
    //+-------------+------+
    //|Time_to_Study|Grades|
    //+-------------+------+
    //|            1|   1.5|
    //|            5|   2.7|
    //|            7|   3.1|
    //|            3|   2.1|
    //|            2|   1.8|
    //|            9|   3.9|
    //|            6|   2.9|
    //|           12|   4.5|
    //|           11|   4.3|
    //|            2|   1.8|
    //|            4|   2.4|
    //|            8|   3.5|
    //|           13|   4.8|
    //|            9|   3.9|
    //|           14|   5.0|
    //|           10|   4.1|
    //|            6|   2.9|
    //|           12|   4.5|
    //|            1|   1.5|
    //|            4|   2.4|
    //+-------------+------+
    //only showing top 20 rows

    //Crie uma matriz de recurso omitindo
    // a ultima coluna
    feature_cols = data.columns[:-1]
    from pyspark.ml.feature import VectorAssembler
    vect_assembler = VectorAssembler
    (inputCols=feature_cols,outputCol="features")

    //Utilize o Assembler criado acima
    //para adicionar a coluna de feicao
    data_w_features = vect_assembler.transform(data)

    // Exibe os dados com colunas adicionais
    //denominadas recursos.
    //Se fosse um problema de regressao linear
    //multipla, voce poderia ver todos os
    // valores de variaveis independentes
    // combinados em uma lista
    data_w_features.show()
    //+-------------+------+--------+
    //|Time_to_Study|Grades|features|
    //+-------------+------+--------+
    //|            1|   1.5|   [1.0]|
    //|            5|   2.7|   [5.0]|
    //|            7|   3.1|   [7.0]|
    //|            3|   2.1|   [3.0]|
    //|            2|   1.8|   [2.0]|
    //|            9|   3.9|   [9.0]|
    //|            6|   2.9|   [6.0]|
    //|           12|   4.5|  [12.0]|
    //|           11|   4.3|  [11.0]|
    //|            2|   1.8|   [2.0]|
    //|            4|   2.4|   [4.0]|
    //|            8|   3.5|   [8.0]|
    //|           13|   4.8|  [13.0]|
    //|            9|   3.9|   [9.0]|
    //|           14|   5.0|  [14.0]|
    //|           10|   4.1|  [10.0]|
    //|            6|   2.9|   [6.0]|
    //|           12|   4.5|  [12.0]|
    //|            1|   1.5|   [1.0]|
    //|            4|   2.4|   [4.0]|
    //+-------------+------+-------
    //only showing top 20 rows

    //Selecione apenas recursos e rotulo do conjunto
    //de dados anterior, pois precisamos dessas duas
    //entidades para construir o modelo de aprendizado
    //de maquina
    finalized_data = data_w_features.select
    ("features","Grades")

    finalized_data.show()

    //+--------+------+
    //|features|Grades|
    //+--------+------+
    //|   [1.0]|   1.5|
    //|   [5.0]|   2.7|
    //|   [7.0]|   3.1|
    //|   [3.0]|   2.1|
    //|   [2.0]|   1.8|
    //|   [9.0]|   3.9|
    //|   [6.0]|   2.9|
    //|  [12.0]|   4.5|
    //|  [11.0]|   4.3|
    //|   [2.0]|   1.8|
    //|   [4.0]|   2.4|
    //|   [8.0]|   3.5|
    //|  [13.0]|   4.8|
    //|   [9.0]|   3.9|
    //|  [14.0]|   5.0|
    //|  [10.0]|   4.1|
    //|   [6.0]|   2.9|
    //|  [12.0]|   4.5|
    //|   [1.0]|   1.5|
    //|   [4.0]|   2.4|
    //+--------+------+
    //only showing top 20 rows

    //Divida os dados em treinamento e modelo
    //de teste com 70% obs. indo em
    //treinamento e 30% em testes
    train_dataset, test_dataset =
    finalized_data.randomSplit([0.7, 0.3])

    //Peek into training data
    train_dataset.describe().show()

    //+-------+------------------+
    //|summary|            Grades|
    //+-------+------------------+
    //|  count|                31|
    //|   mean|3.3451612903225802|
    //| stddev|1.1609877144562206|
    //|    min|               1.5|
    //|    max|               5.0|
    //+-------+------------------+

    //Classe de regressao linear de
    //importacao chamada LinearRegression
    from pyspark.ml.regression import LinearRegression

    //Crie o objeto de regressao linear
    //denominado tendo coluna de feicao
    //como feicoes e coluna de rotulo
    //como Time_to_Study
    LinReg = LinearRegression
    (featuresCol="features", labelCol="Grades")

    //Treine o modelo no treinamento
    // usando o metodo fit ().
    model = LinReg.fit(train_dataset)

    //Preveja as notas usando o metodo evulate
    pred = model.evaluate(test_dataset)

    //Mostrar os valores de Grau previstos
    //ao lado dos valores de Grau reais
    pred.predictions.show()

    //+--------+------+------------------+
    //|features|Grades|        prediction|
    //+--------+------+------------------+
    //|   [1.0]|   1.5|1.5663703357117775|
    //|   [1.0]|   1.5|1.5663703357117775|
    //|   [2.0]|   1.8|1.8366768043045956|
    //|   [3.0]|   2.1|2.1069832728974136|
    //|   [3.0]|   2.1|2.1069832728974136|
    //|   [4.0]|   2.4| 2.377289741490232|
    //|   [4.0]|   2.4| 2.377289741490232|
    //|   [4.0]|   2.4| 2.377289741490232|
    //|   [7.0]|   3.1| 3.188209147268686|
    //|   [7.0]|   3.1| 3.188209147268686|
    //|   [7.0]|   3.1| 3.188209147268686|
    //|   [8.0]|   3.5|3.4585156158615042|
    //|   [8.0]|   3.5|3.4585156158615042|
    //|   [8.0]|   3.5|3.4585156158615042|
    //|   [9.0]|   3.9|3.7288220844543223|
    //|  [10.0]|   4.1|3.9991285530471403|
    //|  [10.0]|   4.1|3.9991285530471403|
    //|  [12.0]|   4.5| 4.539741490232776|
    //|  [13.0]|   4.8| 4.810047958825595|
    //+--------+------+------------------+

    //Descubra o valor do coeficiente
    coefficient = model.coefficients
    print ("The coefficient of the model
    is : %a" %coefficient)
    //The coefficient of the model is
    //: DenseVector([0.2703])

    //Descubra o valor de interceptacao
    intercept = model.intercept
    print ("The Intercept of the model is
     : %f" %intercept)
    //The Intercept of the model is : 1.296064

    //Avalie o modelo usando metricas como
    // erro medio absoluto (MAE), erro
    //quadratico medio (RMSE) e R-quadrado
    from pyspark.ml.evaluation import
    RegressionEvaluator
    evaluation = RegressionEvaluator
    (labelCol="Grades", predictionCol="prediction")

    // Erro de raiz quadrada media
    rmse = evaluation.evaluate
    (pred.predictions, {evaluation.metricName: "rmse"})
    print("RMSE: %.3f" % rmse)

    // Erro Quadrado Medio
    mse = evaluation.evaluate
    (pred.predictions, {evaluation.metricName: "mse"})
    print("MSE: %.3f" % mse)

    // Erro Medio Absoluto
    mae = evaluation.evaluate
    (pred.predictions, {evaluation.metricName: "mae"})
    print("MAE: %.3f" % mae)

    // r2 - coeficiente de determinacao
    r2 = evaluation.evaluate
    (pred.predictions, {evaluation.metricName: "r2"})
    print("r2: %.3f" %r2)
    //RMSE: 0.069
    //MSE: 0.005
    //MAE: 0.056
    //r2: 0.995

    //Crie um conjunto de dados nao rotulado
    //para conter apenas a coluna de recurso
    unlabeled_dataset = test_dataset.select
    ('features')

    Display the content of unlabeled_dataset
    unlabeled_dataset.show()
   // +--------+
   // |features|
   // +--------+
   // |   [1.0]|
   // |   [1.0]|
   // |   [2.0]|
   // |   [3.0]|
   // |   [3.0]|
   // |   [4.0]|
   // |   [4.0]|
   // |   [4.0]|
   // |   [7.0]|
   // |   [7.0]|
   // |   [7.0]|
   // |   [8.0]|
   // |   [8.0]|
   // |   [8.0]|
   // |   [9.0]|
   // |  [10.0]|
   // |  [10.0]|
   // |  [12.0]|
   // |  [13.0]|
   // +--------+

   //Prever a saida do modelo para dados de
   //teste novos e nao vistos usando o metodo
   //transform ()
   new_predictions =
   model.transform(unlabeled_dataset)

   //Display the new prediction values
   new_predictions.show()
   //+--------+------------------+
   //|features|        prediction|
   //+--------+------------------+
   //|   [1.0]|1.5663703357117775|
   //|   [1.0]|1.5663703357117775|
   //|   [2.0]|1.8366768043045956|
   //|   [3.0]|2.1069832728974136|
   //|   [3.0]|2.1069832728974136|
   //|   [4.0]| 2.377289741490232|
   //|   [4.0]| 2.377289741490232|
   //|   [4.0]| 2.377289741490232|
   //|   [7.0]| 3.188209147268686|
   //|   [7.0]| 3.188209147268686|
   //|   [7.0]| 3.188209147268686|
   //|   [8.0]|3.4585156158615042|
   //|   [8.0]|3.4585156158615042|
   //|   [8.0]|3.4585156158615042|
   //|   [9.0]|3.7288220844543223|
   //|  [10.0]|3.9991285530471403|
   //|  [10.0]|3.9991285530471403|
   //|  [12.0]| 4.539741490232776|
   //|  [13.0]| 4.810047958825595|
   //+--------+------------------+
  \end{lstlisting}
\end{quote}
