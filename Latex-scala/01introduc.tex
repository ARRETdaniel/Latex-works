% Prof. Dr. Ausberto S. Castro Vera
% UENF - CCT - LCMAT - Curso de Ci\^{e}ncia da Computa\c{c}\~{a}o
% Campos, RJ,  2020
% Disciplina: Paradigmas de Linguagens de Programa\c{c}\~{a}o
% Aluno:


\chapterimage{ScalaH} % Chapter heading image
\chapter{ Introdu\c{c}\~{a}o}


\begin{quote}



   Scala é a abreviação das palavras SCalable LAnguage. Scala vem com o objetivo de suprir as necessidades dos desenvolvedores de Software modernos. Scala foi iniciada por Martin Odersky em 2001. A sua primeira publicação foi em 20 de janeiro de 2004. Martin, por sua vez, é um professor na  School of Computer and Communication Sciences em Ecole Polytechnique Fédérale de Lausanne (EPFL). É uma linguagem com tipagem estática e dinâmica, orientada a objeto, funcional e de alta performance. Esses fundamentos possibilitam trabalhar em cima de problemas de maneira eficiente e com boa abstração.
   Scala deriva da linguagem de programação Java. Assim sendo, Scala é compilado em “Java-specific formar”  chamado de Java Bytecode.  Java Bytecode por sua vez é uma linguagem abstraída de máquina ou um conjunto de instruções executadas pelo Java Virtual Machine (JVM). Se pode pensar no JVM como um programa que executa outros programas como uma ambiente de máquina virtual  e roda em uma Java Virtual Machine (JVM).
   Apesar das similaridades com Java há muitas diferenças entre as duas linguagens. A linguagem Scala vem com a intenção de superar o Java em suas limitações.
\end{quote}

\begin{quote}


   \hspace{2.5mm} No geral, linguagens de programações são categorizadas nos segundos tipos:
   \begin{itemize}
      \item  High level or Low Level:  Sendo relacionado ao seu nível abstração do mundo real. Essa abstração diz sobre o quão próximo à linguagem de programação trabalha próximo ao que os computadores interpretam.
      \item  Orientação a objeto:  Se baseia no desenvolvimento de programas pensando em seus blocos; objetos. Esses objetos são programados pensando em suas intenções uns com os outros.
      \item  Estático ou Dinâmico: Estético consiste em que os tipos de dados são checados antes do programa ser executado. Dinâmico, por outro lado, os tipos só são conhecidos após o programa estar em execução.
   \end{itemize}


   Ademais, podemos ver que Scala permite Tipagem estática dando a possibilidade de criar aplicações robustas, consertando muitas das falhas presente na linguagem Java. A Orientação a Objeto (POO) é suportada em Scala disponibilizando uma maneira completa de trabalhar de diversas formas com os Objetos criados. Tudo em Scala é um Objeto. Adicionalmente a isso, a programação funcional permite que a linguagem trabalhe com Big Data. Pois há a possibilidade de valores imutáveis, funções sem efeitos colaterais.
   Dessa forma, o com o crescimento da linguagem Scala desde seu lançamento em 2004 com a sua presença crescente nas áreas mais atuais, como Big Data, mostra que é uma linguagem para a atualidade.
   \cite{Elahi}, \cite{Nilanjan}, \cite{Wampler2021}, \cite{Data1},
   \cite{Data2}, \cite{web}, \cite{Whaling}.

\end{quote}


\section{Aspectos hist\'{o}ricos da linguagem Scala}

\begin{quote}
   O Design inicial da Linguagem de programação Scala foi iniciado no ano de 2001, e teve a sua primeira publicação em Janeiro de 2004. Desde seu projeto foi pensada para ser uma linguagem Funcional e Orientada a Objeto.
   Um dos princípios levados em conta nos primeiros passos foi de introduzir uma linguagem derivada do Java, sendo assim algo melhor que o Java. Mas conectando com as suas infraestruturas, logo o JVM e suas bibliotecas.
   Inicialmente, os desenvolvedores da linguagem Scala trabalhavam em uma linguagem chamada de Funnel, que buscava ser uma linguagem de classes puras, e de forma padronizada.
   Apesar de ser uma boa abordagem vista do âmbito acadêmico, mas não foi para os demais.
   Dessa forma, o time de desenvolvedores decidiram recomeçar no desenvolvimento de uma linguagem derivada do Java, que se encontrava no meio de uma linguagem para a comunidade acadêmica como a Funnel, mas também sendo uma linguagem prática. Sem demora, surgiu a linguagem Scala para cumprir esses requisitos.
\end{quote}


\section{\'{A}reas de Aplica\c{c}\~{a}o da Linguagem}
\begin{quote}
   As aplicações da linguagem Scala varia entre o espaço acadêmico, e prático. Sendo assim, se aplica na área de Inteligência Artificial, primordialmente na parte de
   Processamento de Dados, Big Data, Machine Learning. Indo até ao Desenvolvimento Web trabalhando com Scala Js, e Scala Native. Fazendo uso extensivo do JVM e de bibliotecas Java.
\end{quote}
\subsection{ Big Data}
\begin{quote}
   A linguagem Scala é usada no desenvolvimento de software na indústria para Big Data algumas das ferramentas usadas a partir da linguagem Scala e o Apache Spark e Apache Kafka que oferecem diversas APIs. Scala vem com o seu teor estatístico para a análise de dados nesses âmbitos. Entretanto, especialistas da área de Dados no geral por exemplo: Deep Learning(DL), Reinforcement learning (RL), and artificial intelligence (AI) que sao as areas  maior destaque dentro do escopo de Machine Learning (ML). É possível ver que os desenvolvedores desses campos fazem uso majoritariamente da linguagem de programação Python, C + +. Devido a isso o uso dessas linguagens cresceram e o uso da Scala vem caindo comparado com as mesmas. A quantidade de dados produzidos vem crescendo a cada momento, e o processamento de dados para pequenos e grandes volumes de dados estão cada vez mais comuns. Assim para extrair informações valiosas no meio de milhares ou milhões de dados, ferramentas e Frameworks de processamento de dados. Scala se torna um dessas ferramentas para trabalhar com Big Data.
   Scala, diferentemente de umas das linguagens mais usadas para Big Data; Python, se diferencia na necessidade de especificar o seu tipo de dado, sendo que Python não requisita isso.
   Ademais, o objetivo central da linguagem ao trabalhar com BIg Data é alcançar uma solução fácil de processamento, sendo assim com opções de processamento paralelo com pouca mudança no código inicial.
   \cite{Wampler2021},
   \cite{Data1},
   \cite{Data2}.
\end{quote}

\subsection{ Programa\c{c}\~{a}o Web}
\begin{quote}
Scala também é usado para programação Web através do uso conjunto com JavaScript por meio do Scala.js. O Scala ainda está disponível no Scala Native. Atrelado com a sua possibilidade  de programação funcional e Programação Orientada a Objeto (POO) proporciona uma maneira elegante e concisa de desenvolvimento. Como Scala é integrável com Java rodando através da JVM, os programadores usando Scala podem usar bibliotecas de Java de maneira direta, podendo até mesmo chamar códigos Scala a partir do Java.
Por Scala ser uma linguagem de programação Open Source podendo ser usada por qualquer desenvolvedor, diminuindo os custos de desenvolvimento de páginas Web.

\hspace{2.5mm} As vantagens do desenvolvimento Web com a linguagem Scala:
   \begin{itemize}
      \item Alta versatilidade;
      \item Alta performance e código mais enxuto;
      \item Comunidade Open Source;
      \item Vasto conjunto de ferramentas, bibliotecas.

   \end{itemize}

Dessa forma, Scala vem para trazer essas vantagens com um ecossistema bem equipado de Ferramentas para  a várias situações permitindo suporte modular para as necessidades de cada implementação.

\hspace{2.5mm} Ferramentas para o Scala:
   \begin{itemize}
   \item Scalatra: Scalatra é um Framework que é uma forma de se conectar com Ruby Sinatra.  Dando aos desenvolvedores uma forma poderosa de conectar o JVM com o Scala ajudando a criar páginas Web e APIs.
   \item BlueEyes: BlueEyes Framework focado em performance e composição.
   \item Akka HTTP: O Akka HTTP é uma implementação para o Scala pensando para o seu desenvolvimento como se fosse uma idealização de um formato. Traz em seus princípios de não ser uma Framework. Mas mais uma opção de ferramental universal para serviços HTTP. Desse modo, apesar de não ter sido a sua intenção inicial. É visto por muitos como um excelente Framework para o Scala.

   \end{itemize}
   A lista de opções de Framework para a linguagem Scala é imensa, caberá a cada desenvolvedor entender as suas necessidades e buscar entender mais das opções de ferramentas para trabalhar junto ao Scala.


   \cite{Whaling}, \cite{web}.
\end{quote}
\subsection{ outras}
\begin{quote}

   \begin{itemize}
      \item Engenharia de Dados;
      \item Infraestrutura de Sistemas;
      \item Computação distribuída.
   \end{itemize}
\end{quote}
